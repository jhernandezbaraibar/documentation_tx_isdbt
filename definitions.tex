%%%%%%%%%%%%%%%%%%%%%%%%%%%%%%%%%%%%%%%%%%%%%%%%%%%%%%%%%%%%%%%%%%%%%%
% $Id: definitions.tex 973 2012-03-14 18:15:22Z fefo $
%%%%%%%%%%%%%%%%%%%%%%%%%%%%%%%%%%%%%%%%%%%%%%%%%%%%%%%%%%%%%%%%%%%%%%

%%%%%%%%%%%%%%%%%%%%%%%%%%%%%%%%%%%%%%%%%%%%%%%%%%%%%%%%%%%%%%%%%%%%%%
% Definitions
%%%%%%%%%%%%%%%%%%%%%%%%%%%%%%%%%%%%%%%%%%%%%%%%%%%%%%%%%%%%%%%%%%%%%% 
% \newcommand{\TO}[0]{\textbf{to}\xspace}
\newcommand{\Fourier}[1]{\ensuremath{{\mathcal{F}_{#1}}}\xspace}
\newcommand{\Micrograph}[1]{\ensuremath{\mathcal{M}_{#1}}\xspace}
% \newcommand{\PartProj}[2]{\ensuremath{\mathrm{PP}^{#1}_{#2}}\xspace}
\newcommand{\Particle}[1]{\ensuremath{\mathcal{P}_{#1}}\xspace}
\newcommand{\PhaseResidual}[0]{\text{PR}\xspace}
\newcommand{\angstroms}[0]{\text{\AA}\xspace}
\newcommand{\CSP}{\text{CSP}\xspace} %
\newcommand{\CSPNM}[2]{\ensuremath{\CSP_{\text{#1}:\text{#2}}}\xspace} %
\newcommand{\SP}{\text{SP}\xspace} %
\newcommand{\ET}{\text{ET}\xspace} %
\newcommand{\EM}{\text{EM}\xspace} %
\newcommand{\GT}{\text{GT}\xspace} %
\newcommand{\map}[2]{\ensuremath{\mathcal{P}_{#1}^{#2}}\xspace}
%\newcommand{\micrograph}[2]{\ensuremath{\mathcal{M}_{#1}^{#2}}\xspace}
\newcommand{\micrograph}[2]{\ensuremath{M_{#1}^{#2}}\xspace}
\newcommand{\NM}[0]{\ensuremath{{N_\text{M}}}\xspace}
\newcommand{\NP}[0]{\ensuremath{{N_\text{P}}}\xspace}
\newcommand{\parameters}[2]{\ensuremath{\boldsymbol{\Theta}_{#1}^{#2}}\xspace}
% \PP{particle index}{micrograph index}
\newcommand{\PP}[2]{\ensuremath{I^{#2}_{#1}}\xspace}
% \newcommand{\particle}[2]{\ensuremath{\mathcal{P}_{#1}^{#2}}\xspace}
\newcommand{\particle}[2]{\ensuremath{P_{#1}^{#2}}\xspace}
\newcommand{\projection}[3]{\ensuremath{\text{Proj}_{#2}^{#3}({#1})}\xspace}
\newcommand{\reals}[0]{\ensuremath{{\mathcal{R}}}\xspace}
\newcommand{\SNR}{\text{SNR}\xspace}
\newcommand{\stack}[1]{\ensuremath{{\mathcal{S}(#1)}}\xspace}
\newcommand{\tiltangle}[1]{\ensuremath{\alpha_{#1}}\xspace}%{\ensuremath{\alpha_\textsc{ta}^{#1}}\xspace}%
\newcommand{\tiltaxisangle}[1]{\ensuremath{\beta_{#1}}\xspace}%{\ensuremath{\alpha_\textsc{tx}^{#1}}\xspace}
\newcommand{\tiltseries}[1]{\ensuremath{{T}_{#1}}\xspace}
\newcommand{\USP}{\text{USP}\xspace}
\newcommand{\CC}{\text{CC}\xspace}
\newcommand{\cost}[1]{\ensuremath{\nu\left(#1\right)}\xspace}
\newcommand{\Cost}[2][]{
  \ifthenelse{\isempty{#1}}%
    {\ensuremath{\bar\nu_{#2}}\xspace}% if #1 is empty
    {\ensuremath{\bar\nu_{#2}\left(#1\right)}\xspace}% if #1 is not empty
}
\newcommand{\weight}[1]{\ensuremath{w_{#1}}\xspace}
\newcommand{\degrees}[0]{\ensuremath{^{\circ}}\xspace}
\newcommand{\etal}[0]{\text{et al.}\xspace}
\newcommand{\average}[1]{\langle{#1}\rangle\xspace}
\newcommand{\unit}[1]{\text{#1}} 
\newcommand{\fdef}[1]{\emph{#1}} 
\newcommand{\fnote}[1]{{{\color{blue} \S#1\S}}} 
\newcommand{\snote}[1]{\marginpar{\raggedright
    {\hspace{0pt}\textsf{\color{Maroon}\small #1}}}}
\newcommand{\inlinefrac}[2]{\ensuremath{^{#1}\!/\!_{#2}}} 
\newcommand{\fpref}[2]{\ref{#1}\subref*{#2}}

%% \newcommand{\unappendix}{\par
%%   % \setcounter{chapter}{0}%
%%   % \setcounter{section}{0}%
%%   \gdef\@chapapp{\chaptername}%
%%   \gdef\thechapter{\@arabic\c@chapter} 
%% }

\newcommand{\confidential}[1]{ %
  \ifthenelse{\boolean{isConfidential}}{#1}{ %
    {\color{Blue}El texto correspondiente, ha sido suprimido por
      razones de confidencialidad solicitada por una de las
      partes. Una vez hecha la publicaci\'on de este trabajo, el
      contenido se har\'a p\'ublico.%
    }%
  }%
}
% Put the marginpar in the correct (outer) side of the page; this
% should be temporary, just for the revisions.
\let\oldmarginpar\marginpar
\renewcommand\marginpar[1]{\-\oldmarginpar[\raggedleft $\RHD$]%
  {\raggedright $\LHD$}}

%%%%%%%%%%%%%%%%%%%%%%%%%%%%%%%%%%%%%%%%%%%%%%%%%%%%%%%%%%%%%%%%%%%%%%%%%%
% Revisions mode
%%%%%%%%%%%%%%%%%%%%%%%%%%%%%%%%%%%%%%%%%%%%%%%%%%%%%%%%%%%%%%%%%%%%%%%%%%
% \newcommand{\new}[1]{{\marginpar{}\color{Blue}{#1}}}
\newcommand{\new}[1]{#1}
\newcommand{\answer}[2]{%<--Important for the blank spaces
  \ifthenelse{\equal{#1}{1}}{{\color{ForestGreen}{#2}}}{}%<--
  \ifthenelse{\equal{#1}{2}}{{\color{BurntOrange}{#2}}}{} }


\def\DOT{.}
\def\model#1#2{\ensuremath{\mathcal{M}_{#1}^{#2}}\xspace}
\def\norm#1{\ensuremath{\left\|#1\right\|}}
\def\projection#1#2#3{\ensuremath{\mathcal{P}_{#2}^{#3}{#1}}}
\def\refeq#1{(\ref{#1})}
\def\etal{\textit{et al.}}
\def\curve{\ensuremath{\mathcal{C}}}
\def\image{\ensuremath{\mathcal{I}}}
\def\dif{\ensuremath{\mathrm{d}}}
\def\reals{\ensuremath{\mathbb{R}}}
\def\FigExt{png} % Bigger files, slower compilation, best resolution.
\def\FigExt{jpg} % Smaller files, faster compilation, lower resolution.
% \def\FigExt{eps} % Postscript 


% %%% Originales de tesina.cls
\newcommand{\sigla}[1]{{\uppercase{#1}}\xspace}
% \newcommand{\puntos}[1]{\textsf{[#1 pts.]}}
% \newcommand{\esperanza}[1]{I\!\!E\left\{#1\right\}}
% \newcommand{\fourier}[1]{I\!\!F\left\{#1\right\}}
% \newcommand{\invfourier}[1]{I\!\!F^{-1}\left\{#1\right\}}
% \newcommand{\conv}[2]{#1\ast #2}
% \newcommand{\ddd}[0]{tridimensional}%{\small 3}\textsc{d}}
% \newcommand{\dd}[0]{bidimensional}%{\small 2}\textsc{d}}
% \newcommand{\figcaption}[1]{\textsl{#1}}
% \newcommand{\figtitle}[1]{\textbf{#1}}
% \newcommand{\dsi}{\sigla{dsi}}
% \newcommand{\gcp}{\sigla{gcp}}
% \newcommand{\ddt}{\mbox{$\frac{\rd{}}{\rd{t}}$}}
% \newcommand{\dpdpt}[1]{\frac{\partial{#1}}{\partial{t}}}
% \newcommand{\traspuesta}[1]{#1^\top}
% \newcommand{\sfgap}{\hspace{0.05\figurewidth}} % espacio entre subfigures
\newcommand{\tesina}[0]{tesis}
% \newcommand{\nmbr}[1]{\oldstylenums{#1}}
% \newcommand{\extraparsep}[0]{\hspace{.45cm}}
% \newcommand{\figurewidth}{\textwidth} % ancho de las figuras
% \newcommand{\decision}{
%     \begin{array}{c}
%     \scriptstyle{\mathcal{X}^m\rightarrow O}\vspace{-0.14cm}\\%
%     \gtrless\vspace{-0.1cm}\\%
%     \scriptstyle{\mathcal{X}^m\rightarrow B}
%     \end{array}
% }
% \newcommand{\ecuacionref}[1]{(\ref{#1})}
% \newcommand{\gracias}[2]{\textsf{#1} #2\\}
% \newcommand{\ingles}[1]{\textsl{#1}}
% \newcommand{\modelo}[1]{\textsl{#1}}
% \newcommand{\textfig}[1]{\textsf{#1}}
% \newcommand{\sustituirsf}[1]{\psfrag{#1}[][][1]{\textsf{#1}}}
% \newcommand{\sustituir}[1]{\psfrag{#1}[][][1]{#1}}
% \newcommand{\reales}{\mbox{$I\!\!R$}}
% \newcommand{\rd}[1]{\mbox{$\,\,\mathrm{d}{#1}$}}
% \renewcommand{\chaptermark}[1]{\markboth{\thechapter.\ #1}{}}
