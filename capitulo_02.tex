\chapter{Fundamento Teórico}

La sección de canal van a dar pie a los time interleavers y esas cosas, no? O sea, hablar de fading tanto en frecuencia y tiempo y cómo contrarrestarlo?
Yo pondría la parte de DFT, FFT y convolución circular dentro de OFDM. En FFT van a hablar del algoritmo? Me parece medio al santo botón.
En la parte de códigos les faltó hablar de convolucionales.


\section{Modelado del canal}
	\subsection{Canales Continuos}
	\subsection{Canales Discretos}
	\subsection{Las no idealidades del Canal}
No idealidades del canal inalámbrico móvil.

\section{Estrategias para mitigar los efectos del canal}

	\subsection{C\'odigos de detecci\'on y correci\'on de errores}

La comunicación entre emisor y receptor puede modelarse mediante el proceso de la Figura \ref{diagrama_codificacion}. La situación es la siguiente, una fuente emisora envía mensajes $m$ (palabras fuente) al receptor a través de un canal de comunicación. El mensaje debe ser traducido a algún mensaje que el canal esté capacitado para enviar, estos mensajes se conocen como palabras código.

\begin{figure}
\centering
\includegraphics[scale=0.45]{figuras/cap02/diagrama_codificacion}
\caption{\label{diagrama_codificacion} Esquema básico de codificación de canal.}
\end{figure}

Al otro lado del canal llega un mensaje codificado $c'$, el cual seguramente sea erróneo, pues en todo proceso real de comunicación existe ruido e imperfecciones en los canales. El mensaje es decodificado en una palabra $m'$, y generalmente $m' \neq m$.

Se desea que el receptor sea capaz de darse cuenta si el mensaje $m'$ es realmente lo que se transmitió del otro lado, y más aún, poder corregirlo. 

La Teoría de Códigos es un campo de la matemática aplicada que busca resolver los problemas de las etapas de codificación-decodificación y corrección, y que presenta su propia complejidad.

La transmisión inalámbrica de una señal la expone a diversas fuentes de ruido, con lo cual los tipos de errores generados pueden ser muy variados. Por ejemplo los errores en ráfaga, en los que un conjunto de bits consecutivos se ven alterados, son muy comunes en las comunicaciones inalámbricas. También podría suceder que el canal radioeléctrico presente distorsión en algunas portadoras en particular.

El estándar ISDB-T hace uso de distintas técnicas modernas para la protección de los datos en transmisión. De hecho para proteger los datos en los ejemplos mencionados el estándar utiliza la \textit{dispersión de energía} y el  \textit{entrelazamiento frecuencial}. Para la comprensión del estándar y el desarrollo de \textit{gr-isdbt-tx}, es importante conocer el funcionamiento de estas técnicas. Profundizar en estos temas escapa los objetivos de este trabajo, por lo cual los detalles técnicos se pueden encontrar en las bibliografías mencionadas.

Para asegurarse que el receptor pueda llevar a cabo satisfactoriamente la demodulación y decodificación en una transmisión jerárquica, en la cual se utilizan múltiples parámetros de transmisión, se utiliza una señal denominada Transmission and Multiplexing Configuration Control (TMCC).

Como se verá en el Capítulo XX, la TMCC junto con otras señales piloto y las señales correspondientes a la transmisión de los datos útiles, conforman el cuadro OFDM.

Al tratarse de una señal que contiene información crítica sobre la transmisión se la debe proteger fuertemente frente a los distintos tipos de errores que podría sufrir durante su transmisión.

En particular ISDB-T establece que para la TMCC se debe utilizar el código acortado (200,118) del \textit{difference-set cyclic code} (273,191) como código corrector de errores.

\subsection{Códigos Cíclicos}
El conjunto $GF(2) \triangleq \{0,1\}$, con las operaciones de suma $" + "$ y producto $" \times "$ usuales módulo 2, cumple con la propiedad de que cualquier elemento de $GF(2)$ distinto de cero tiene inverso. Esta propiedad se cumple trivialmente en este conjunto y es la condición necesaria para que $GF(2)$ sea un \textit{Campo de Galois}. Es común encontrar que a este campo también se lo llame \textit{campo binario} y se lo denote como $\mathbb {F}_2$.
Las operaciones de suma y producto definidas en $GF(2)$ son asociativas, conmutativas y distributivas, y llevan elementos de $GF(2)$ en elementos de $GF(2)$. Por esto $GF(2)$ también es un \textit{anillo}. 
El conjunto de todos los polinomios con coeficientes en $GF(2)$ con las operaciones usuales de suma y producto forman un \textit{anillo de polinomios} en $GF(2)$ y se denota como $GF(2)[x]$. Por ejemplo $g(x) = x^3 + x + 1$ es un elemento de $GF(2)[x]$.

Sea $\textbf{c} = (c_0, c_1, ..., c_{n-1}) \in GF(2)$, con $GF(2)$ tal como se describió anteriormente. Un código $\mathcal{C}$ de bloque $(n, k)$ se dice que es un \textit{código cíclico} si para cada vector $\textbf{c} = (c_0, c_1, ..., c_{n-1}) \in \mathcal{C}$ cualquier rotación circular a la derecha de $C$ también pertenece a $\mathcal{C}$, es decir $(c_{n-1}, c_0, c_1, ..., c_{n-2}) \in \mathcal{C}$.
Los códigos de bloque se caracterizan por codificar mensajes de longitud fija $k$ en \textit{codewords} de longitud fija $n$, con lo cual el tamaño del mensaje original se incrementa en $n-k$.
Cada \textit{codeword} del código $\mathcal{C}$ puede ser representada en una forma polinomial de la siguiente manera:

\begin{equation}
c(x) = \sum_{i = 0}^{n-1}c_i x^i
\end{equation}

A continuación se enumera una serie de propiedades de los códigos cíclicos, en \cite{moon2005error} se puede encontrar una demostración detallada de cada una de ellas.

\begin{itemize}
\item{Un código cíclico es un código lineal de bloque}
\item{Cada \textit{codeword} se corresponde con un polinomio}
\item{Los polinomios del código forman un \textit{ideal} en $GF(2)[x]/(x^n-1)$}
\item{Para un código cíclico existe un generador $g(x)$ que es divisor de $x^n-1$ y que puede generar todos las \textit{codewords} $c(x)=m(x)g(x)$}
\end{itemize}

Se puede probar que esto implica la existencia de una \textit{matriz de chequeo de paridad} $\mathbb{H} \in \mathcal{M}_{(n-k)\times n}$ tal que para toda \textit{codeword} \textbf{c} de $\mathcal{C}$ se cumple $\textbf{c} \mathbb{H} ^T = \textbf{0}$.


El proceso de codificación se realiza de la siguiente manera, primero se construye el polinomio $x^{n-k}m(x)$ de grado $n$. Luego se divide entre el polinomio generador $g(x)$ y el resto de esa división es el polinomio de paridad $d(x)$ que se le agregará al mensaje:

\begin{equation}
x^{n-k}m(x) - q(x)g(x) = d(x)
\end{equation}

La \textit{codeword} se forma de la siguiente manera:

\begin{equation}
c(x) = x^{n-k}m(x)-d(x) = q(x)g(x)
\end{equation}

Como se trata de un múltiplo de $g(x)$, entonces efectivamente es una \textit{codeword} válida. La representación vectorial de la \textit{codeword} queda de la siguiente manera:

\begin{equation}
\textbf{c} = (-d_0, -d_1, ..., -d_{n-k-1}, m_0, m_1, ..., m_{k-1})
\end{equation}

En una situación en la que se recibe una palabra $\textbf{r}$ cuyo mensaje es $\textbf{m}$  y sus bits de paridad son $\textbf{d}$, el procedimiento para detectar si hubo error es codificar el mensaje $\textbf{m}$ que se recibió con el mismo codificador utilizado por el transmisor (ambas partes deben conocer el polinomio generador), y luego comparar el $\textbf{d'}$ obtenido con el $\textbf{d}$ recibido. Si ambos difieren entonces hubo error. 
Por ejemplo, para un código cíclico (7, 4) con polinomio generador $g(x) = x^3 + x + 1$ se desea codificar el mensaje 1001. Los mensajes codificados tendran $n-k = 7 - 4 = 3$ bits de paridad. El mensaje en su forma polinomial queda $m(x) = 1 + x^3$.
Los bits de paridad se obtienen calculando el resto de la division $x^{(7-4)}m(x)/g(x)$, los coeficientes de ese resto seran los bits de la paridad buscada. Operando se llega a que la paridad es 011 y el mensaje codificado queda 0111001.

	
	
	\subsection{C\'odigos BCH}
	\subsection{C\'odigos de Reed-Solomon}
	\subsection{Entrelazamiento de datos}

\section{Modulación OFDM}

Chang presenta un metodo para lograr la multiplexion canales de datos a través de un medio de frecuencia acotada, en \cite{chang-ofdm}, que elimina los efectos de interferencia intersimbolica e intercanal. Implico un cambio importante en la teoría de telecomunicaciones de la época, pues hasta entonces, los resultados existentes tomaban como funciones modulantes ortogonales, señales limitadas en el tiempo, lo que implica grandes anchos de banda en frecuencia, y en los canales de banda acotada implementados en la practica se traducían en interferencias producto de los recortes en banda. 

En el paper, Chang postula la idea de una nueva clase de funciones modulantes acotadas en frecuencia, que ademas, permite modular de manera independiente amplitud y frecuencia. En ese momento se sentaban las bases de la modulacion OFDM (Orthogonal Frecuency-Division Multiplexing).

Uno de los mayores problemas de los sistemas FDM, era la incapacidad para escalar en cantidad de canales. La complejidad y el costo de construir los osciladores para la cantidad de portadoras necesarias, mantenían la brecha entre la teoría y la practica. 

La solución a estos problemas, llega cuando se logra programar computadoras capaces de procesar grandes cantidades de datos, mediante la implementación del algoritmo de “Fast Fourier Transform”. 

Weinstein y Ebert, resolvieron en \cite{discrete-ofdm} el problema de la escalabilidad de los sistemas FDM, mediante la conjugacion de los mencionados avances tecnologicos, discretizando las señales a transmitir, y modulandolas por computadora, en lugar de usar los bancos de osciladores. 

Una señal multitonal, puede ser vista como la transformada de fourier de un tren de pulsos, y la demodulacion coherente a su vez puede entenderse como la aplicación en tiempo continuo de una transformada inversa de Fourier. Entonces, probaron que muestreando la señal de origen, y mediante la implementación de un modem sobre una computadora que ejecute el algoritmo de la transofrmada rápida de fourier, se pueden obtener aproximaciones suficientemente cercanas a los de la señal original. 

En principio el resultado es valido para un sistema FDM con N canales simultáneos, con portadoras separadas en frecuencia en distancias suficientemente cercanas, como para aproximar la respuesta al impulso del canal, como si fuese de modulo constante sobre cada uno de los N canales. 

Pero falta un paso mas, pues la hipótesis del comportamiento del canal constante, se aleja bastante de la realidad. Plantearon entonces un canal, de respuesta al impulso lineal en frecuencia.

En estas condiciones, y atendiendo ademas que la señal discreta a transmitir solo “vive” en el ancho de banda de transmisión (N.df), consideraron a la señal a transmitir como una transformada de fourier enventanada. Desarrollando estas ideas lograron probar que, si la ventana es plana en las frecuencias de interés, y se agregan “guardas” de seguridad a ambos lados de las portadoras activas, de modo que las colas del enventanado caigan de forma continua, (alejando el enventanado de la idealidad de las ventanas rectangulares) se logran condiciones para reducir la distorsión generada por la respuesta al impulso del canal, a efectos transitorios de dispersión rápida, y la señal en recepción sigue convergiendo a la señal transmitida.

\section{MPEG y sus Estandares}

El Moving Picture Experts Group (MPEG)\cite{MPEG} es un grupo de trabajo conformado por expertos internacionales, formado por la Organizacion Internacional de Normalizacion (ISO) en conjunto con la Comision Electrotecnica Internacional (IEC), con el objetivo de desarrollar estandares para la codificacion, compresion y transmision de audio y video.

Uno de los estandares publicados por el MPEG, es MPEG-4. Consta de metodos para la compresion digital de contenidos audiovisuales,  y abarca la difusion de los mismos a traves de una amplia gamas de tecnologias, desde el streaming de datos a traves de la web, codificacion de voz y video para telefonia y videoconferencias, comercializacion de discos compactos (CD) y hasta formatos para la transmision de Television. 

Es en este ultimo punto donde se vincula con ISDB-T Internacional, puesto que para la codificacion de fuente en la norma, fue seleccionado el Estandar MPEG-4 Parte 10 “Advanced Video Coding”, tambien denominado H.264.

Para garantizar que los receptores de television digital ISDB-T, tambien sean compatibles con los transmisores tanto de ISDB-T como de ISDB-T International, se encapsulan los videos codificados en H.264 dentro de un formato denominado “Transport Stream” que se define en la norma MPEG-2 Parte 1 – Sistemas.

En el transmisor gr-isdbt-tx, tomamos como fuente de datos un archivo codificado como Transport Stream, para garantizar esta compatibilidad.
	\subsection{MPEG 2 Transport Stream}
	

Un Paquetized Elementary Stream (PES), es una especificacion de MPEG 2 para el transporte de flujos elementales, generalmente las salidas del codificadores de audio y video. En ISDB-T, los elementary Streams pueden ser videos, audios en mas de un idioma, archivos de subtitulo, grillas de programacion y tablas de informacion de transmision.

Un Transport Stream (TS) es un contenedor de datos en el que se encapsulan en conjunto un PES, junto con codigos de correccion de errores y flags de sincronismo. La combinacion de estos datos, permite mantener la continuidad de la transmision cuando el canal se degrada, respondiendo a su propia naturaleza. 

En ISDB-T, se multiplexan varios TS al inicio de la cadena de transmisión, para crear un unico TS de transmision. El mismo sera sometido luego a codificaciones de canal, para robustecerlo aun mas frente a las perdidas del canal. Este proceso se discutira luego en el capitulo 3.

La estructura de un Transport Stream es la siguiente:
(Imagen)

Los datos de los elementary streams se recortan en secciones de 188 bytes, este tamaño tan chico, permite que se realice un entrelazamiento con otros ES con muy baja latencia, y con una mayor resistencia ante las perdidas.

	\subsection{Tablas PMT}

Dentro de los Transport Streams se define el concepto de Programas. Cada programa esta definido en una tabla denominada PMT (Program Map Table), que viaja multiplexada en el TS de transmisión, identificada por un PID único. Los Elementary Streams asociados con el programa en cuestión, tienen sus PIDs listados en la PMT. En general, se asocia cada canal con un programa, aunque también podrían utilizarse programas para (completar)

Cuando un receptor decide reproducir un canal en particular, lo que tiene que hacer es decodificar los payloads contendidos en los TS cuyos PIDs están en la tabla PMT

Ademas de la tabla PMT, existen otros tipos de tablas en MPEG-2. Para el alcance de este documento, nos interesa detallar solo dos mas.
La Program Asociation Table (PAT), contiene una lista con todos los programas 



	
	\subsection{Tablas PAT}
	The program association table lists all programs available in the transport stream. Each of the listed programs is identified by a 16-bit value called program number. Each of the programs listed in PAT has an associated value of PID for its program map table (PMT). 
	The value 0x0000 for program number is reserved to specify the PID where to look for network information table. If such a program is not present in PAT the default PID value (0x0010) shall be used for NIT. 
	TS packets containing PAT information always have PID 0x0000. 
	\subsection{Paquetes Nulos}
	Null packets
	Some transmission schemes, such as those in ATSC and DVB, impose strict constant bitrate requirements on the transport stream. In order to ensure that the stream maintains a constant bitrate, a Multiplexer may need to insert some additional packets. The PID 0x1FFF is reserved for this purpose. The payload of null packets may not contain any data at all, and the receiver is expected to ignore its contents. 

