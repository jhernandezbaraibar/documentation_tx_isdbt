\chapter{Fundamento Teórico}

La sección de canal van a dar pie a los time interleavers y esas cosas, no? O sea, hablar de fading tanto en frecuencia y tiempo y cómo contrarrestarlo?
Yo pondría la parte de DFT, FFT y convolución circular dentro de OFDM. En FFT van a hablar del algoritmo? Me parece medio al santo botón.
En la parte de códigos les faltó hablar de convolucionales.


\section{Modelado del canal}
	\subsection{Canales Continuos}
	\subsection{Canales Discretos}
	\subsection{Las no idealidades del Canal}
No idealidades del canal inalámbrico móvil.

\section{Estrategias para mitigar los efectos del canal}

	\subsection{C\'odigos de detecci\'on y correci\'on de errores}
	\subsection{C\'odigos de Reed-Solomon}
	\subsection{C\'odigos BCH}
	\subsection{Entrelazamiento de datos}

\section{Modulación OFDM}

	\subsection{Fundamentos}
		\subsubsection{Transformada de Fourier}
		\subsubsection{Convoluci\'on circular}
		\subsubsection{Esquema b\'asico}
	
	\subsection{Continuos}

	\subsection{Discretos}

\section{El estandard MPEG-4}
	\subsection{Generalidades}
	\subsection{Transport Stream Packet}
	\subsection{Tablas PAT}
	\subsection{Tablas PMT}

