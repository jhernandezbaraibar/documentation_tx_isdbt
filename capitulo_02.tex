\chapter{Fundamento Teórico}

La sección de canal van a dar pie a los time interleavers y esas cosas, no? O sea, hablar de fading tanto en frecuencia y tiempo y cómo contrarrestarlo?
Yo pondría la parte de DFT, FFT y convolución circular dentro de OFDM. En FFT van a hablar del algoritmo? Me parece medio al santo botón.
En la parte de códigos les faltó hablar de convolucionales.

\section{Canal}
No idealidades del canal inalámbrico móvil.

\section{Transformada de Fourier (DFT, FFT) y convolución circular}

\section{Sistemas OFDM}

\subsection{Continuos}

\subsection{Discretos}
Se puede ampliar bastante el desarrollo que hace Pablo sobre sistemas discretos. 

\section{Códigos de detección y correción de errores (Reed-Solomon, BCH)}


\section{Flujos de transporte MPEG-2}
\subsection{Paquete de flujo de transporte (Transport Stream Packet)}
\subsection{PAT}
\subsection{PMT}

