\chapter{Conclusiones y trabajo a futuro}

Este proyecto tenia como objetivos, dos grandes ítems. Por un lado buscábamos implementar un transmisor de televisión digital basado en SDR, cuya validez estaría determinada por la capacidad de transmitir hacia un televisor comercial homologado por LATU. Por otro lado, también buscábamos echar luz sobre el conocimiento disponible en el país sobre la norma ISDB-T. 

En cuanto al primer objetivo, entendemos que ha sido cumplido de forma satisfactoria. Hemos probado el transmisor contra varios televisores comerciales, de diferentes proveedores, y en todos los casos el equipo encuentra el canal transmitido de forma exitosa, y decodifica las tres capas procesadas de forma correcta. 

Durante las pruebas encontramos diversos problemas que, mediante distintas estrategias, hemos podido solucionar. El problema de fluidez de reproducción del video fué quizás el más difícil de detectar, pero también su causa era tan sencilla como la incorrecta utilización de los parámetros. Cada prueba realizada aportó su cuota en el proceso de validación del transmisor. Como se mencionó anteriormente, dentro del flowgraph las cosas parecían funcionar perfectamente pero al pasar a testear sobre \textit{gr-isdbt} aparecieron otra clase de problemas, al igual que en los televisores comerciales. Esto deja en evidencia la necesidad de tener un profundo dominio de cada aspecto de la norma, de otra manera, por ejemplo, un simple error al realizar los ajustes de atraso hace que el sistema deje de funcionar por completo.

Por otro lado, sobre el objetivo de aumentar la cantidad y la calidad de la información sobre la norma disponible para los técnicos nacionales, consideramos que este objetivo ha sido cumplido, si bien el alcance del mismo no estará determinado hasta que este documento se haga público. El repositorio con el proyecto está disponible de forma gratuita para cualquier interesado en aprender mas de la norma ISDB-T, y en conjunto con gr-isdbt, se tiene una herramienta de análisis del sistema, que cuenta con control absoluto de todas las variables. 

Esperamos que este proyecto pueda colaborar con el mejor desempeño de la red nacional de transmisoras de televisión abierta, y que se haga uso de las herramientas que la norma incluye. En esa línea, esperamos aumentar la popularidad del repositorio publicando a la brevedad un video demostrativo en YouTube sobre nuestro transmisor, que esperamos que logre atraer interesados al código que puedan realizar más pruebas y nuevos aportes para mejorar la calidad del mismo.

Queda pendiente, para este grupo de trabajo, el mejor desarrollo de algunos de los bloques del sistema. Algunos ítems, como la portabilidad del divisor jerárquico, será necesario resolver cuanto antes, para robustecer el funcionamiento del transmisor. También las mejoras por el lado de la optimización de la programación y el manejo de recursos de la PC. En la primer ejecución de \textit{gr-isdbt} para transmitir con el USRP, el bloque OFDM Frame Structure acaparaba todo el procesador. Para seguir adelante fué necesario replantearse la programación del bloque, de otra manera era imposible tener un transmisor funcional. Esto se logró y mejoró notablemente el rendimiento, pero aún queda trabajo en algunos otros bloques demandantes.

Finalmente, nos gustaría plantear un objetivo a mediano o largo plazo, que es un pequeño proyecto que hubiese sido implementado de disponer con el tiempo suficiente. Este transmisor, necesita como fuente de datos un archivo de BTS. Sería muy interesante crear un bloque TS Remux que tome tres flujos MPEG-2 y devuelva el BTS con su correspondiente patrón de ordenamiento. De todas maneras, utilizando los bloques existentes en GNU Radio, es posible alimentar al \textit{gr-isdbt-tx} con tres flujos MPEG-2 y lograr un funcionamiento satisfactorio.