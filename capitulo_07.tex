\chapter{Conclusiones y trabajo a futuro}

Este proyecto tenia como objetivos, dos grandes items. Por un lado buscábamos implementar un transmisor de televisión digital basado en SDR, cuya validez estaría determinada por la capacidad de transmitir hacia un televisor comercial homologado por LATU. Por otro lado, también buscábamos echar luz sobre el conocimiento disponible en el país sobre la norma ISDB-T. 

En cuanto al primer objetivo, entendemos que ha sido cumplido de forma satisfactoria. Hemos probado el transmisor contra dos televisores comerciales, de diferentes proveedores, y en ambos casos el equipo encuentra el canal transmitido de forma exitosa, y decodifica las tres capas procesadas de forma correcta. 

Si bien la señal se decodifica correctamente, cumpliendo con la calidad de video esperada en cada capa, sí existe un problema de continuidad leve, tanto en video como en el audio. Estos problemas aparecen sólo cuando se prueba gr-isdbt-tx con el USRP, ya sea orientado a un televisor comercial o a otra computadora ejecutando gr-isdbt. Cuando la prueba de recepción se realiza sobre el mismo flowgraph, el problema no aparece, por lo que entendemos que el bug se refiere a un problema del USRP o de su configuración.

Dado que el equipo tiene una variabilidad del clock no despreciable, y ademas no es capaz de soportar la tasa de muestras exacta que espera la norma (8.127 MS/s contra 8.000 MS/s del equipo) esperamos a la brevedad detectar la fuente del bug siguiendo estas ideas y resolverlo lo antes posible.

Por otro lado, sobre el objetivo de aumentar la cantidad y la calidad de la información sobre la norma disponible para los técnicos nacionales, consideramos que este objetivo ha sido cumplido. El repositorio con el proyecto esta disponible de forma gratuita para cualquier interesado en aprender mas de la norma ISDB-T, y en conjunto con gr-isdbt, se tiene una herramienta de análisis del sistema, que cuenta con control absoluto de todas las variables. 

Esperamos que este proyecto pueda colaborar con el mejor desempeño de la red nacional de transmisoras de televisión abierta, y que se hagan uso de las herramientas que la norma incluye. 

Queda pendiente, para este grupo de trabajo, el mejor desarrollo de algunos de los bloques del sistema. Algunos items, como la portabilidad del divisor jerárquico, sera necesario resolver cuanto antes, para robustecer el funcionamiento del transmisor. Otros, como la ya mencionada resolución de los problemas de implementación con el USRP, serán resueltos a la brevedad, no solo para dar cumplimiento al objetivo principal de este proyecto, sino también, para vencer la constante curiosidad que caracteriza a los ingenieros.

Finalmente, nos gustaría plantear un objetivo a mediano o largo plazo, que es un pequeño proyecto que hubiese sido implementado de disponer con el tiempo suficiente. Este transmisor, necesita como fuente de datos un archivo en formato MPEG TS. Seria muy interesante crear un conversor de MPEG 4 a transport streams, de modo que el transmisor pueda  enviar al televisor cualquier video proveniente de la memoria del PC. Esta funcionalidad seria muy interesante y extendería el potencial del transmisor, ya que para este proyecto, obtener archivos en formato TS nos fue bastante complicado.