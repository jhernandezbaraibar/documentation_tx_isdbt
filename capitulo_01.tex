\chapter{Introducción}

TV digital, SDRs, gr-isdbt, evolucion de la TV digital en el Uruguay,......

Este capítulo es muy importante, y hay que poner la motivación y la importancia del proyecto. Hay que ver si no me olvido de nada, pero lo que sí o sí hay que poner son la didáctica (mostrar un sistema punta a punta bastante complejo como ISDB-T, que no es una radio AM digamos) y la posibilidad de armar una transmisora de TV con unos cientos de dólares. Esto último hay que verificarlo en la parte de evaluación: sobre todo el alcance. 

Motivacion, facultad busca darle al estado conocimiento sobre ISDBT, software de codigo abierto y el alcance global de la informacion bajo este paradigma. Complementar el trabajo del Receptor.

Miren todo lo que aprendí, ver~\cite{Autor}.

