\chapter{Introducción}

En Mayo de 2001, la ARIB (Association of Radio Industries and Businesses)\cite{ARIB} presento la primera version de su estandard para la transmision de television digital. Coloquialmente denominada "Norma Japonesa de Television Digital", academicamente ISDB-T, por sus siglas en ingles provenientes de \textit{Integrated Services Digital Broadcasting, Terrestrial}, la norma sintetiza un conjunto de requerimientos técnicos para la utilización eficiente del espectro radioeléctrico para la transmisión de datos multimedia, con la colaboración y el aval de todos los actores de la industria.

Esta norma basa muchos de sus conceptos en la norma DVB-T, publicada por primera vez en el año 1997 por la organización europea DVB (Digital Video Broadcasting). La posterioridad de ISDB-T con respecto a esta, permitió que se robustecieran algunos de los aspectos mas criticados de la norma europea, resultando en un estandard mas robusto para la transmisión.

Actualmente, existen en el mundo cuatro grandes estandares comerciales. Ademas de ISDB-T y DVB-T, estan la norma china DTMB (Digital Terrestrial Multimedia Broadcast) y la norma norteamericana de la ATSC (Advanced Television Systems Comitee). La elección de que norma adoptar por parte de los gobiernos nacionales, radica exclusivamente en decisiones políticas, que escapan a los objetivos de este documento.

En Uruguay, al igual que en gran parte de Latinoamerica, se adopto en 2011 una version de ISDB-T denominada ISDB-T International, la cual es a grandes rasgos identica a la primera, salvo por algunos cambios menores como el cambio en la codificación de fuente (pasa del estandard MPEG-2 a MPEG-4) y la elección de otro estandard de interactividad (se cambia de BML a Ginga)

Luego de la adopción del estandard, se fijo para el año 2015 como fecha limite para el denominado "apagon analogico", fecha en la cual se dejaría de transmitir televisión por vías analógicas, pasando exclusivamente a medios digitales, liberando los espectros asignados para los canales de TV abierta para otros fines. 

Durante la implementación del marco legal de la nueva norma de televisión digital, se entregaron 22 licencias para transmisión de contenidos bajo la norma ISDB-T International (En mas, ISDB-T por simplicidad). Al día de hoy, tres años después de la fecha limite para el apagón, solo algunos de los actores del rubro están brindando el servicio de forma adecuada y llama la atención la baja participación del sector comercial en la transformación analógico-digital. 

Relevamos las situación de las estaciones de transmisión de televisión digital a lo largo del país, y hemos encontrado que la cobertura abarca solo una parcialidad del territorio nacional, existiendo incluso departamentos del interior del país que aun no tienen cobertura en su totalidad.

\begin{figure}
	\centering
	\includegraphics[scale=0.3]{figuras/cap01/mapa_estaciones}
	\caption{\label{mapa_estaciones} Distribución de las estaciones transmisoras de TV digital en el Uruguay.}
\end{figure}

La situación de los consumidores del servicio, también dista de la ideal, pues, la televisión analógica sigue siendo la mayor puerta de acceso al medio. La Encuesta Continua de Hogares del Instituto Nacional de Estadística\cite{ine2017}, cuyos indicadores son una muestra representativa de la situación de todos los hogares del país, dió a conocer que en el año 2017 solamente el 47 \% de los hogares encuestados tienen recepción de TV digital abierta. Tanto es así, que el mencionado apagón analógico, fue pospuesto por tiempo indeterminado. El alto costo del recambio de equipamiento, y posturas sobre la democratización del acceso a la información para personas de bajos recursos, fundamentan estas decisiones.

Es en ese contexto nacional que la Facultad de Ingeniería, quien luego de la adopción legal de la norma se puso como objetivo la apropiación tecnológica de la norma, nos encomendó el desarrollo de un transmisor de televisión digital, basado en SDR (Radio Definida por Software) y de código abierto.

Nuestro proyecto complementa el trabajo iniciado por Pablo Flores, Maria Simon y Federico Larroca, quienes en 2016 publicaron \textit{gr-isdbt} \cite{gr-isdbt}, un receptor de televisión digital, también de código abierto y bajo el paradigma de SDR. Ambos trabajos en conjunto, permiten visualizar y ayudan a comprender el funcionamiento del sistema de transmisión digital de televisión de punta a punta, teniendo acceso completo a todo lo que sucede en el mismo en cualquier punto de la cadena de transmisión. 

Ademas, aparece la posibilidad de recrear una planta de transmisión de televisión a muy bajo costo, permitiendo su reproducción tanto en el hogar por entusiastas, en el aula por docentes o en la industria, por técnicos, lo cual puede colaborar con el mejoramiento de la calidad del servicio actual. A su vez, puede servir como ejemplo para algunos de los cursos de la Facultad, generalmente catalogados por el estudiantado como muy teóricos y con poco alcance práctico. 

Implementar un transmisor de televisión digital, no es una tarea sencilla. El primer problema a enfrentar es el acceso a la información técnica. Existe poca documentación generada en el país, para cumplir con las condiciones técnicas de un sistema complejo y que, además, ya lleva 7 años de vigencia como oficial. La norma presentada por la ARIB deja varias zonas grises, asume por conocidos conceptos clave, y no se explaya mas de lo necesario en cuestiones de fondo. 

Ademas, existen fuertes limitaciones económicas para hacerse con software o hardware comercial que resuelvan incluso algunas de las funcionalidades mas básicas que exige la norma. 

Esta tesis intenta suplir esa carencia, en principio complementando el trabajo iniciado por el grupo ARTES con el receptor \textit{gr-isdbt}. Se desarrolló a lo largo de este proyecto, un transmisor de televisión digital que cumple con las condiciones establecidas en la norma, y cuyas señales son decodificables por los televisores comerciales homologados por el LATU.

Contar con el trabajo presentado en \textit{gr-isdbt} fue de una ayuda mayúscula, ya que basarse en el paradigma de código abierto ayudo a comprender, sintetizar y testear los conceptos teóricos vertidos en la norma, lo que fue fundamental para la comprensión de las funcionalidades que sería necesario implementar para poder transmitir. 

Para este grupo de trabajo, es importante destacar lo valioso de la existencia de proyectos de código abierto, incontables veces encontramos en la comunidad puntos de vista, ideas y hasta algoritmos para resolver los problemas encontrados en el camino. Es por eso que esperamos poder contribuir con ella, poniendo a disposición de cualquier persona el transmisor \textit{gr-isdbt-tx}, para que continúen con el trabajo de aprendizaje y la optimización del mismo por técnicos y estudiantes, seguramente con un mejor panorama del rubro, que el que tuvimos al implementar este proyecto.

Esperamos también, mediante el desarrollo de este documento, poder contribuir con la comunidad nacional de técnicos que trabajan en el rubro, y que no cuentan con documentación técnica generada por y para la norma nacional, con los problemas y las particularidades que la transmisión tiene en nuestro país y no tener que abstraer de trabajos de terceros, que resolvieron problemas similares en contextos diferentes. Entendemos que en el mismo, los conceptos desarrollados por la norma se bajan a tierra en un formato de código abierto, y gratuito, lo que democratiza el acceso a una información a la que hoy por hoy solo se accede por medio de hardware y software propietario con licencias de costos elevados.

Para esta documentación, que acompaña el código presentado para el transmisor, definimos seis capítulos en los que se explica el desarrollo del mismo. En el capitulo 2 presentamos algunos de los conceptos fundamentales de telecomunicaciones sobre los que se construye la norma. El capitulo 3 realiza un breve pasaje por los puntos clave del sistema transmisor ISDB-T, los cuales son necesarios para comprender algunos de los bloques que conforman el sistema. Para profundizar mas en los mencionados conceptos, invitamos al lector a revisar la tesis de maestría de Pablo Flores, que pueden encontrar en \cite{gr-isdbt}. El capitulo 4 se detiene particularmente en el concepto de radio definida por software y  presenta en detalle una implementacion del mismo, en particular aquel sobre el cual se desarrollo el transmisor, que es GNURadio. Luego en el capitulo 5 analizamos punto por punto el código generado para implementar el transmisor, explicando en cada paso los conceptos del capitulo 2 y 3 que se necesita aplicar en cada bloque, y como se extrapolaron a C++, lenguaje en el que se escribió cada uno de los bloques de procesamiento. Mas adelante, en el capitulo 6 mostramos el desempeño del transmisor como un todo, realizando las evaluaciones practicas del mismo en función de los objetivos de este proyecto y se comentan los resultados obtenidos. Para terminar, en el capitulo 7, presentamos las conclusiones del proyecto en particular y planteamos algunos desafíos que seria interesante afrontar en un futuro.






