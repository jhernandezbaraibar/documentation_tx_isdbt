\chapter*{Resumen}
\addcontentsline{toc} {chapter} {Resumen}%

Han pasado 7 años desde que Uruguay adopto el estandard ISDB-T como la norma nacional para la transmisión de televisión digital. Encontramos que al día de hoy, el estado de la red dista bastante de lo proyectado cuando se firmo el decreto. El acceso a la información técnica se ve dificultado por la poca cantidad de documentos alusivos al alcance. Ademas, para la formación de técnicos en el área, es indispensable contar con hardware muy costoso y con distintos softwares propietarios, de difícil acceso tanto para los interesados.

Nos proponemos en este trabajo aportar desde nuestro lugar en el rubro, mas información sobre la norma y su implementación, respaldando los conocimientos teóricos adquiridos con una solución técnica accesible para todos. 

Es por eso que, complementando el trabajo iniciado por el grupo ARTES de la Facultad de Ingeniera en 2016 con el proyecto gr-isdbt, presentamos un transmisor de televisión digital, implementado completamente en el formato de código abierto y amparado en el paradigma de Radio Definida por Software. 

De este modo, cualquier interesado en entender la norma sera capaz de poner en funcionamiento un sistema de transmisión de televisión digital a pequeña escala. La conjunción entre gr-isdbt y gr-isdbt-tx, permite trabajar con el sistema completo de punta a punta, teniendo acceso absoluto a toda la información en transmisión, en cualquier punto de la cadena.