\chapter*{Resumen}
\addcontentsline{toc} {chapter} {Resumen}%

La radio definida por software es un paradigma donde los sistemas de radiocomunicaciones, tradicionalmente implementados en hardware, son en cambio implementados por software. Esto típicamente se lleva a cabo a través de una computadora personal encargada de ejecutar el software, y un hardware genérico que se encarga de sintonizar, filtrar y muestrear la señal de radio a cierta tasa. De esta manera la señal puede ser procesada por la computadora.
Hay una gran variedad de estos equipos que van desde unos pocos dólares hasta algunos miles de dólares. El paradigma de la radio definida por software permite lograr una implementación completa de sistemas de radiocomunicaciones de manera económica y sencilla en cierto sentido.

GNU Radio es un toolkit abierto y libre, ampliamente utilizado para la implementación de radios definidas por software. Provee bloques de procesamiento de señal de propósito general como filtros, re-samplers, moduladores y demoduladores analógicos y digitales. Estos bloques pueden ser combinados para lograr diferentes sistemas de comunicación. También es posible la implementación de nuevos bloques personalizados por parte de los usuarios, de manera de contribuir con el código fuente de GNU Radio. Esto hace que la comunidad de GNU Radio sea muy activa y resulte en una contribución continua de los usuarios tanto en el desarrollo de nuevas radios definidas por software o el testeo de otras implementaciones.

Integrated Services for Digital Broadcasting-Terrestrial (ISDB-T), es el estándar japonés para televisión digital terrestre que ha sido adoptado por la mayoría de los países en América del Sur. En Uruguay mediante un decreto de Presidencia, se adoptó ISDB-T como la norma nacional para la transmisión de televisión digital. Esto implicó la necesidad de desarrollar profesionales locales con un profundo dominio técnico sobre la norma, dando lugar también a importantes trabajos de investigación. Uno de esos trabajos es gr-isdbt, un receptor abierto de ISDB-T basado en radios definidas por software.

En este trabajo presentamos el primer transmisor de ISDB-T abierto, basado en radios definidas por software e implementado sobre GNU Radio. Se logró complementar el trabajo realizado en el proyecto gr-isdbt, para lograr una implementacion de un sistema de transmisión de televisión digital de punta a punta. La implementación lograda en gr-isdbt-tx \cite{gr-isdbt-tx} es capaz de modular por completo una señal ISDB-T fullseg, lo que significa que es capaz de multiplexar en frecuencia hasta tres flujos de transporte MPEG con múltiples programas de audio y video al mismo tiempo. Cada flujo de transporte es codificado de manera independiente logrando diferentes grados de robustecimiento de la señal, formando lo que se denomina capa. Esto se consigue utilizando modulación OFDM y dividiendo el espectro en 13 grupos de portadoras idénticos, denominados segmentos. Cada capa cuenta con su propia codificación de canal y modulación, que es independiente del resto de las capas.

Una vez que la señal ISDB-T es conformada, se transmite por el aire y puede ser demodulada por cualquier dispositivo compatible con la norma, por ejemplo un TV comercial, un set top box ó una PC con gr-isdbt. Esto convierte a una PC, junto con un hardware SDR con posibilidad de transmitir, en un relativamente económico transmisor de TV digital flexible y totalmente configurable. De hecho quienes posean las licencias adecuadas podrían tener una estación transmisora de TV de manera relativamente sencilla. La aplicación resulta particularmente interesante para la televisión comunitaria en donde el aspecto económico es quizás la restricción más importante.

De manera alternativa, es posible simular todo el sistema concatenado transmisor-receptor dentro de GNU Radio. Esto genera una variada gama de posibilidades de su uso, por ejemplo puede ser utilizado con propósitos educativos mostrando de manera práctica cómo funciona un sistema de comunicación complejo. También podría ser utilizado para evaluar el desempeño de la codificación de las distintas capas jerárquicas al atravesar el canal, o evaluar el impacto de señales interferentes que podrían aparecer en el canal. 

Daremos una descripción paso a paso de todos los puntos que hubo que resolver para lograr este transmisor, centrándonos en los aspectos claves que son necesarios dominar para conseguir una implementación exitosa. También describiremos las pruebas llevadas a cabo así como sus resultados. 