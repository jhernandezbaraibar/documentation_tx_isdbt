\chapter*{Resumen}
\addcontentsline{toc} {chapter} {Resumen}%

Han pasado 7 años desde que Uruguay adoptó mediante un decreto de Presidencia, el estandard ISDB-T como la norma nacional para la transmisión de televisión digital (TVD). Al día de hoy, el despliegue de la red de transmision de TVD se encuentra en un estado bastante mas primitivo de lo proyectado cuando se firmó el decreto. Quizás uno de los factores causantes de esta situación, es que el acceso a la información técnica se ve dificultado por la poca cantidad de documentos alusivos al alcance publico, en particular de información generada en el país. Ademas, para la formación de buenos técnicos en el área, es indispensable contar con equipos de hardware muy costoso y con distintos softwares propietarios, de difícil acceso para los interesados.

Nos proponemos en este trabajo aportar desde nuestro lugar en el rubro, mas información sobre la norma y su implementación, respaldando los conocimientos teóricos adquiridos con una solución técnica accesible para todos. 

Es por eso que, complementando el trabajo publicado por el grupo ARTES de la Facultad de Ingeniera en 2016, en lo que fuera el proyecto gr-isdbt, presentamos un transmisor de televisión digital, implementado completamente en el formato de código abierto y amparado en el paradigma de Radio Definida por Software. 

De este modo, cualquier interesado en entender la norma sera capaz de poner en funcionamiento un sistema de transmisión de televisión digital a pequeña escala. La conjunción entre gr-isdbt y gr-isdbt-tx, permite trabajar con el sistema completo de punta a punta, teniendo acceso absoluto a toda la información en transmisión, en cualquier punto de la cadena.