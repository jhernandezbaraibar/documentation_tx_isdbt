\section{Pruebas y Resultados}

\subsection{Pruebas en ambiente controlado}
\begin{frame}{Pruebas en ambiente controlado}
\begin{block}{Caso ideal, conexion directa}
	\begin{itemize}
		%\subsection{ISDB-T (International Standard for Digital Broadcasting - Terrestrial)}	
		\item { Se decodificaoron las tres capas exitosamente }
		\item { Verificamos la funcionalidad basica del sistema }
	\end{itemize}
\end{block}

\begin{block}{Caso ruidoso, simulamos perdidas en el canal}
	\begin{itemize}
		%\subsection{ISDB-T (International Standard for Digital Broadcasting - Terrestrial)}	
		\item {	Observamos el efecto de los bloques correctores de errores }
		\item { Encontramos un umbral de ruido tolerable }
	\end{itemize}
\end{block}
\end{frame}

\subsection{Pruebas en canal real}

\begin{frame}{Pruebas en canal real}
\begin{block}{Pruebas contra gr-isdbt en otra PC}
	\begin{itemize}
		%\subsection{ISDB-T (International Standard for Digital Broadcasting - Terrestrial)}	
		\item { Encontramos  }
		\item { Verificamos la funcionalidad basica del sistema }
	\end{itemize}
\end{block}

\begin{block}{Pruebas contra equipo comercial Rohde-Schwarz}
	\begin{itemize}
		%\subsection{ISDB-T (International Standard for Digital Broadcasting - Terrestrial)}	
		\item {	Observamos la constelación recibida, detectamos un bug importante }
		\item { Primeras diferencias notorias entre gr-isdbt-tx y los transmisores comerciales }
	\end{itemize}
\end{block}

\begin{block}{Pruebas contra televisor comercial}
	\begin{itemize}
		%\subsection{ISDB-T (International Standard for Digital Broadcasting - Terrestrial)}	
		\item { Comprobamos que funcionan las tres capas correctamente  }
		\item { Se cumple con los objetivos planteados al principio del proyecto }
	\end{itemize}
\end{block}
\end{frame}